\documentclass{beamer}

% configure for german language
\usepackage[ngerman]{babel}   % allow german special characters

% for pictures
\usepackage{graphicx}

% for code-snippets
\usepackage{listingsutf8}
\usepackage{color}

% set theme and color scheme
\usetheme{Dresden}
%\usetheme{CambridgeUS}
%\usecolortheme{beaver}

% special character set
\usepackage{pifont}

% document data
\title{Dynamc compilation}
\author{Fabian Gruber}

%gets rid of bottom navigation bars
\setbeamertemplate{footline}[page number]{}

%gets rid of navigation symbols
\setbeamertemplate{navigation symbols}{}

\begin{document}
%%%%%%% TITLE %%%%%%%%%%%%%%%%%%%%%%%%%%%%%%%%%%%%%%%%%%%%%%%%%%%%%%%%%%%%%%%%%%%%%%%%%%%%%%%%%%%%%%%%%%%%%%%%%%%%%%%%%%
	%% name and author paper
	\frame{\titlepage}

%%%%%%% GOALS %%%%%%%%%%%%%%%%%%%%%%%%%%%%%%%%%%%%%%%%%%%%%%%%%%%%%%%%%%%%%%%%%%%%%%%%%%%%%%%%%%%%%%%%%%%%%%%%%%%%%%%%%%

	\section{Introduction}

	\begin{frame}
		\frametitle{Goals}

		\begin{itemize}
			\itemsep12pt

			\item Write a JIT for a very simple Lisp like language
			\item There already is an interpreter for that language
			\item The JIT and the interpreter should interoperate
		\end{itemize}
	\end{frame}

%%%%%%% LANGUAGE INTRO %%%%%%%%%%%%%%%%%%%%%%%%%%%%%%%%%%%%%%%%%%%%%%%%%%%%%%%%%%%%%%%%%%%%%%%%%%%%%%%%%%%%%%%%%%%%%%%%%

	\section{Language}

	\begin{frame}
		\begin{center}
			\Large{The language}
		\end{center}
	\end{frame}

	\begin{frame}
		\frametitle{Reader syntax}

		\begin{tabular}{lcl}
			integers               & & 1, 5, 23, $1\_000\_000$                                                               \\[9pt]
			floating point numbers & & 1.5, 3.14, 5., .7                                                                     \\[9pt]
			symbols                & & true, false, $+$, $*$                                                                 \\[9pt]
			strings                & & "hello world"                                                                         \\[9pt]
			characters             & & $\backslash a$, $\backslash b$, $\backslash c$, $\backslash tab$ $\backslash newline$ \\[9pt]
			lists                  & & (), (1 2 3), (1 (2 3) 4)                                                              \\[9pt]
			quotation              & & '5, '"foo", 'symbol, '(a b c)                                                         \\[9pt]
		\end{tabular}
	\end{frame}

	\begin{frame}
		\frametitle{Special forms}

		\begin{tabular}{lcl}
			function application  & & (+ 5 6)                        \\[9pt]
			quotation             & & (quote (1 2 3)), (quote 5)     \\[9pt]
			if-then-else          & & (if true 5 7)                  \\[9pt]
			expression sequence   & & (do (println "hi") 5)          \\[9pt]
			variable binding      & & (let  (a 5) (b 6)      (+ a b) \\[9pt]
			                      & & (let* (a 5) (b (+ a 1) (+ a b) \\[9pt]
			lambda expression     & & (lambda (a b c) (* a (+ b c))) \\[9pt]
			global variable       & & (define pi 3.14)               \\[9pt]
		\end{tabular}

	\end{frame}

	\begin{frame}
		\frametitle{Refs}

		\begin{itemize}
			\itemsep12pt

			\item All data types are immutable
			\item Except refs
			\item (define foo (ref))
			\item (set foo 42)
			\item (get foo) $\rightarrow$ 42
		\end{itemize}
	\end{frame}

%%%%%%% THE COMPONENTS %%%%%%%%%%%%%%%%%%%%%%%%%%%%%%%%%%%%%%%%%%%%%%%%%%%%%%%%%%%%%%%%%%%%%%%%%%%%%%%%%%%%%%%%%%%%%%%%%

	\section{Overview}

%	\begin{frame}
%		\begin{center}
%			\Large{Main components}
%		\end{center}
%	\end{frame}

	\begin{frame}
		\frametitle{Main components}

		\begin{itemize}
			\itemsep12pt

			\item \textbf{Reader}
			\item \textbf{Analyzer}
			\item \textbf{Interpreter/JIT}
		\end{itemize}
	\end{frame}

	\begin{frame}
		\frametitle{Reader}

		\begin{itemize}
			\itemsep12pt
		
			\item Parses strings/files into S-expressions
			\item Simple hand written recursive descent parser
		\end{itemize}
	\end{frame}

	\begin{frame}
		\frametitle{Analyzer}

		\begin{itemize}
			\itemsep12pt
		
			\item 'Parses' S-expressions into AST
			\item Performs some simple checks on the source
			\item Also gathers some type information
		\end{itemize}
	\end{frame}

	\begin{frame}
		\frametitle{Interpreter/JIT}

		\begin{itemize}
			\itemsep12pt
		
			\item Meta circular interpreter evaluates expressions into values
			\item When functions are called often enough the JIT is invoked
			\item The JIT performs some more analysis and optimization
		\end{itemize}
	\end{frame}

%%%%%%% VALUE LAYOUT %%%%%%%%%%%%%%%%%%%%%%%%%%%%%%%%%%%%%%%%%%%%%%%%%%%%%%%%%%%%%%%%%%%%%%%%%%%%%%%%%%%%%%%%%%%%%%%%%%%

	\section{Memory layout}

%	\begin{frame}
%		\begin{center}
%			\Large{Main components}
%		\end{center}
%	\end{frame}

	\begin{frame}
		\frametitle{Values}

		\begin{itemize}
			\itemsep12pt
		
			\item All values are allocated on the heap
			\item Every value starts with a 8 byte tag.
			\item The tag identifies a values type and a functions arity
		\end{itemize}
	\end{frame}

	\begin{frame}
		\frametitle{Functions}

		For every function there is an object containing
		\begin{itemize}
			\itemsep12pt
		
			\item a pointer to the executable code,
			\item a pointer to the functions AST,
			\item a counter for how often the function has been called
		\end{itemize}
	\end{frame}

	\begin{frame}
		\frametitle{Closures}

		Functions can be nested, thus we need to support closures.

		Closure objects contains:
		\begin{itemize}
			\itemsep12pt
		
			\item A type/arity tag (like any other object)
			\item A pointer to the executable code
			\item A pointer to the shared function object
			\item An array of the values the closure captures
		\end{itemize}
	\end{frame}

%%%%%%% THE JIT %%%%%%%%%%%%%%%%%%%%%%%%%%%%%%%%%%%%%%%%%%%%%%%%%%%%%%%%%%%%%%%%%%%%%%%%%%%%%%%%%%%%%%%%%%%%%%%%%%%%%%%%

	\section{The JIT}

	\begin{frame}
		\frametitle{Features of the JIT}
	
		\begin{itemize}
			\itemsep12pt
		
			\item Tail call removal for self calls
			\item Inlining of small functions
			\item Remove null \& type check before function calls in simple cases
		\end{itemize}
	\end{frame}

%%%%%%% DEMO %%%%%%%%%%%%%%%%%%%%%%%%%%%%%%%%%%%%%%%%%%%%%%%%%%%%%%%%%%%%%%%%%%%%%%%%%%%%%%%%%%%%%%%%%%%%%%%%%%%%%%%%%%%

	\section{Demo}

	\begin{frame}
		\Large
		\begin{center}
			A short Demo
		\end{center}
	\end{frame}

%%%%%%% SUMMARY %%%%%%%%%%%%%%%%%%%%%%%%%%%%%%%%%%%%%%%%%%%%%%%%%%%%%%%%%%%%%%%%%%%%%%%%%%%%%%%%%%%%%%%%%%%%%%%%%%%%%%%%

	\section{Summary}

	\begin{frame}
		\Large
		\begin{center}
			Summary
		\end{center}
	\end{frame}

	\begin{frame}
		\begin{center}
			\begin{itemize}[<+->]
				\item[] Have I met all my goals?
				\item[] Not everything worked out as planned

				\begin{itemize}[<+->]
					\itemsep12pt
	
					\item Write a JIT for a very simple Lisp like language
						\begin{itemize}[<+->]
							\item[] Ok
						\end{itemize}					
					\item There already is an interpreter for that language 
						\begin{itemize}[<+->]
							\item[] Didn't gather enough information, it had to be rewritten
						\end{itemize}
					\item The JIT and the interpreter should interoperate   
						\begin{itemize}[<+->]
							\item[] The JIT can only call compiled functions 
							\item[] $\Rightarrow$ functions are compiled on demand
						\end{itemize}
				\end{itemize}
			\end{itemize}
		\end{center}
	\end{frame}

	\begin{frame}
		\begin{center}
			Other problems \\

			\vspace*{18pt}

			\begin{itemize}[<+->]
%				\itemsep12pt

				\item I Planned to use LLVM as my code generator
					\begin{itemize}[<+->]
						\item[] Too heavy weight for this small project
					\end{itemize}
				\item[]
				\item Now I use libjit
					\begin{itemize}[<+->]
						\item[] libjit's tail call optimizer is broken
						\item[] It seems that lijit is no longer being developed
						\item[] Had to repair it
					\end{itemize}
				\item[]
%				\item I planned to do a points to analysis and allocate values on the stack
%					\begin{itemize}[<+->]
%						\item[] Ran out of time
%					\end{itemize}
			\end{itemize}
		\end{center}
	\end{frame}

	\begin{frame}
		\frametitle{Trivia}

		\begin{center}
			\vspace*{18pt}

			\begin{itemize}
				\itemsep12pt

				\item ~5000 lines of C++ code
				\item Requires a C++11 compiler
				\item ~50 lines of C code for libjit patch
				\item Only external dependency is libjt
			\end{itemize}
		\end{center}
	\end{frame}


%%%%%%% WRAP IT UP %%%%%%%%%%%%%%%%%%%%%%%%%%%%%%%%%%%%%%%%%%%%%%%%%%%%%%%%%%%%%%%%%%%%%%%%%%%%%%%%%%%%%%%%%%%%%%%%%%%%%

	\section{}

	\begin{frame}
		\begin{center}
			\Large{Thank you for your attention}
		\end{center}
	\end{frame}

	\begin{frame}
		\begin{center}
			\Large{Questions?}
		\end{center}
	\end{frame}

%	\section{Referenzen}
%
%	\begin{frame}		
%		Dawson R. Engler, \\
%		\textbf{VCODE}: a retargetable, extensible, very fast dynamic code generation system. \\
%		 In Proceedings of the ACM SIGPLAN 1996 conference on Programming language design and implementation,
%		 pages 160-170, 1996.
%
%		\vspace*{12pt}
%
%		Chris Lattner and Vikram Adve, \\
%		\textbf{LLVM}: A Compilation Framework for Lifelong Program Analysis \& Transformation. \\
%		In Proceedings of the international symposium on Code generation and optimization: feedback-directed and runtime optimization,
%		pages 75-86, 2004.
%
%		\vspace*{12pt}
%
%		The LLVM project \\
%		\url{http://www.llvm.org}
%	\end{frame}
	
\end{document}

